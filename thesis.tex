\documentclass[
  paper=a4, % DIN-A4-Format
  11pt, % Schriftgröße
  bibliography=totoc, % Bibliografie automatisch im Inhaltsverzeichnis generieren
  parskip=off, % Absatzabstand: off, half, full
  oneside, % einseitiges Layout
%  twoside, % zweiseitiges Layout
  article, % verwendet KOMA-Klasse scrartcl
%  longdoc=true,
  accentcolor=tud2b,
%  colorbacktitle, % Hintergrund des Titels einfärben
  colorback, % Hintergrund unter dem Titel einfärben
  type=bsc, % für Bachelorarbeit
]{tudthesis}

\usepackage[T1]{fontenc}
\usepackage[utf8]{inputenc} % korrekte Darstellung von Umlauten u. Sonderzeichen
\usepackage[stable]{footmisc} % mehr Optionen für Fußnoten
\usepackage{booktabs} % hübschere Tabellen als der LaTeX-Standard
\usepackage{multirow} % Create tabular cells spanning multiple rows
\usepackage{longtable} % große Tabellen, die sich über mehrere Seiten erstrecken
\usepackage{listings} % bietet Umgebung, um Programmiercode zu setzen inkl. Syntaxhighlighting
\usepackage{amsmath} % Setzen mathematischer Formeln
\usepackage{ellipsis} % korrigiert Aussehen von „…“
\usepackage{enumitem} % mehr Optionen für Aufzählungen
\usepackage[shortcuts]{extdash} % mehr Optionen für Worttrennung
\usepackage{setspace} % Zeilenabstände verändern: \singlespacing, \onehalfspacing oder \doublespacing
\usepackage{lipsum} % zur Erzeugung von Lorem-Ipsum-Blindtext
\usepackage[math]{blindtext} % zur Erzeugung von deutschem Blindtext
\usepackage{hyperref} % Verlinkungen im Dokument
\usepackage{csquotes}

% INFO %
% Das hyperref-Paket sollte möglichst als letztes geladen werden, darum steht es weiter hinten in der Präambel.
% Leider funktioniert hyperref nicht 100%-ig einwandfrei mit tudstyle; bei Problemen damit lieber aus dem Dokument entfernen!
% Einstellungen für hyperref
\hypersetup{%
    pdftitle=Motion R-CNN,
    pdfauthor=Simon Meister,
    pdfsubject=B.Sc. Thesis,
    unicode=true, % benötigt, damit Umlaute im pdftitle richtig dargestellt werden
    breaklinks=true
    %% colorlinks=true
    %% linkcolor=NavyBlue,
    %% urlcolor=NavyBlue,%Green4,
    %% citecolor=NavyBlue%DeepPink3
}

\usepackage[
    backend=biber, % biber ist das Standard-Backend für Biblatex. Für die Abwärtskompatibilität kann hier auch bibtex oder bibtex8 gewählt werden (siehe biblatex-Dokumentation)
    style=numeric, %numeric, authortitle, alphabetic etc.
    autocite=footnote, % Stil, der mit \autocite verwendet wird
    sorting=ynt, % Sortierung: nty = name title year, nyt = name year title u.a.
    sortcase=false,
    url=false,
    hyperref=auto,
    giveninits=true,
    maxbibnames=100
]{biblatex}

\renewbibmacro*{cite:seenote}{} % um zu verhindern, dass in Fußnoten automatisch "(wie Anm. xy)" eingefügt wird
\DeclareFieldFormat*{citetitle}{\mkbibemph{#1\isdot}} % zitierte Titel kursiv formatieren
\DeclareFieldFormat*{title}{\mkbibemph{#1\isdot}} % zitierte Titel kursiv formatieren

\addbibresource{bib.bib} % Hier Pfad zu deiner .bib-Datei hineinschreiben
\nocite{*} % Alle Einträge in der .bib-Datei im Literaturverzeichnis ausgeben, auch wenn sie nicht im Text zitiert werden. Gut zum Testen der .bib-Datei, sollte aber nicht generell verwendet werden. Stattdessen lieber gezielt Einträge mit Keywords ausgeben lassen (siehe \printbibliography in Zeile 224).


% INFO %
% Hier kannst du deine persönlichen Daten hineinschreiben und diese Variablen verwenden, damit du es nicht jedes Mal neu schreiben musst.
\newcommand*{\myname}{Simon Meister}
\newcommand*{\mytitlede}{Motion R-CNN: 3D-Bewegungsschätzung auf Instanzebene mit Regionsbasierten CNNs}
\newcommand*{\mytitleen}{Motion R-CNN: Instance-level 3D Motion Estimation with Region-based CNNs}
\newcommand*{\mymail}{\href{mailto:simon.meister@stud.tu-darmstadt.de}{simon.meister@stud.tu-darmstadt.de}}
\newcommand*{\myprof}{Prof. Stefan Roth, Ph.D.}
\newcommand*{\myinstitute}{Visual Inference Group}
\newcommand*{\myfaculty}{Fachbereich Informatik}

%%%%%%%%%%%%%%%%%%%%%%%%%%%%%%%%%%%%%%%%%%%%%%%%%%%%%%%%%%%%%%%%%%%%%%%%%%%%%%%%

\begin{document}
\parindent 0em % Erstzeileneinzug


\newcommand{\todo}[1]{\textbf{\textcolor{red}{#1}}}
\setlength{\belowrulesep}{0pt}

% Titelei
\author{\myname}
\thesistitle{\mytitleen}{\mytitlede}
\birthplace{Erbach}
\date{23.11.2017}
\referee{\myprof}{M.Sc. Junhwa Hur}
\department{\myinstitute}
\group{\myfaculty}
\dateofexam{23.11.2017}{23.11.2017}
\makethesistitle

% Eigenständigkeitserklärung: muss nach \makethesistitle erscheinen, sonst wird sie als erste Seite des Dokuments gesetzt.
\affidavit[23.11.2017]{\myname}
%\affidavit{\myname}

\pagestyle{myheadings} % Seitenstil umschalten
\mymarkright{Version: \today} % Inhalt der Fußzeile


\begin{abstract}

% Many state of the art energy-minimization approaches to optical flow and scene
% flow estimation rely on a rigid scene model, where the scene is
% represented as an ensemble of distinct, rigidly moving components, a static
% background and a moving camera.
% By constraining the optimization problem with a physically sound scene model,
% these approaches enable state-of-the art motion estimation.

With the advent of deep learning, it has become popular to re-purpose
generic deep networks for classical computer vision problems involving
pixel-wise estimation.

Following this trend, many recent end-to-end deep learning approaches to optical
flow and scene flow predict complete, high resolution flow fields with
a generic network for dense, pixel-wise prediction, thereby ignoring the
inherent structure of the underlying motion estimation problem and any physical
constraints within the scene.

We introduce a scalable end-to-end deep learning approach for dense motion estimation
that respects the structure of the scene as being composed of distinct objects,
thus combining the representation learning benefits and speed of end-to-end deep networks
with a physically plausible scene model inspired by slanted plane energy-minimization approaches to
scene flow.

Building on recent advances in region-based convolutional networks (R-CNNs),
we integrate motion estimation with instance segmentation.
Given two consecutive frames from a monocular RGB-D camera,
our resulting end-to-end deep network detects objects with precise per-pixel
object masks and estimates the 3D motion of each detected object between the frames.
By additionally estimating a global camera motion in the same network,
we compose a dense optical flow field based on instance-level and global motion
predictions. We train our network on the synthetic Virtual KITTI dataset,
which provides ground truth for all components of our system.

\end{abstract}

\renewcommand{\abstractname}{Zusammenfassung}
\begin{abstract}

Mit dem Aufkommen von Deep Learning
ist die Umfunktionierung generischer Deep Networks ein
beliebter Ansatz für klassische Probleme der Computer Vision geworden,
die pixelweise Schätzung erfordern.

Viele aktuelle end-to-end Deep Learning Methoden
für optischen Fluss oder Szenenfluss folgen diesem Trend und berechnen
vollständige und hochauflösende Flussfelder mit generischen
Netzwerken für dichte, pixelweise Schätzung, und ignorieren damit die
inhärente Struktur des zugrundeliegenden Bewegungschätzungsproblems und jegliche physikalische
Randbedingungen innerhalb der Szene.

Wir stellen ein skalierbares end-to-end Deep Learning Verfahren für dichte
Bewegungschätzung vor,
das die Struktur einer Szene als Zusammensetzung eigenständiger
Objekte respektiert, und kombinieren damit die Repräsentationskraft und Geschwindigkeit
von end-to-end Deep Networks mit einem physikalisch plausiblen Szenenmodell,
das von slanted-plane Energieminimierungsmethoden für Szenenfluss inspiriert ist.

Hierbei bauen wir auf den aktuellen Fortschritten in regionsbasierten Convolutional
Networks (R-CNNs) auf und integrieren Bewegungsschätzung mit Instanzsegmentierung.
Bei Eingabe von zwei aufeinanderfolgenden Frames aus einer monokularen RGB-D
Kamera erkennt unser end-to-end Deep Network Objekte mit pixelgenauen Objektmasken
und schätzt die 3D-Bewegung jedes erkannten Objekts zwischen den Frames ab.
Indem wir zusätzlich im selben Netzwerk die globale Kamerabewegung schätzen,
setzen wir aus den instanzbasierten und globalen Bewegungsschätzungen ein dichtes
optisches Flussfeld zusammen.
Wir trainieren unser Netzwerk auf dem synthetischen Virtual KITTI Datensatz,
der Ground Truth für alle Komponenten unseres Systems bereitstellt.


\end{abstract}

\clearpage

%\pagenumbering{Roman} % Seitennummerierung auf römische Zahlen ändern

\setcounter{tocdepth}{3}
\tableofcontents
\setcounter{page}{1}
\clearpage

% Aktuelle Seitenzahl speichern, da Wechsel auf arabische Zahlen die Zählung zurücksetzt
%% \newcounter{savedromanpagenumber}
%% \setcounter{savedromanpagenumber}{\value{page}}
%\pagenumbering{arabic} % Arabische Seitenzahlen

\section{Introduction}
\parindent 2em
\onehalfspacing

\subsection{Motivation}

% introduce problem to sovle
% mention classical non deep-learning works, then say it would be nice to go end-to-end deep

% Steal intro from behl2017 & FlowLayers

Deep learning research is moving towards videos.
Motion estimation is an inherently ambigous problem and
A recent trend is towards end-to-end deep learning systems, away from energy-minimization.
Often however, this leads to a compromise in modelling as it is more difficult to
formulate a end-to-end deep network architecture for a given problem than it is
to state a fesable energy-minimization problem.
For this reason, we see lots of generic models applied to domains which previously
employed intricate physical models to simplify optimization.
On the on hand, end-to-end deep learning may bringe unique benefits due do the ability
of a learned system to deal with ambiguity.
On the other hand,
%Thus, there is an emerging trend to unify geometry with deep learning by
% THE ABOVE IS VERY DRAFT_LIKE

Recently, SfM-Net \cite{SfmNet} introduced an end-to-end deep learning approach for predicting depth
and dense optical flow in monocular image sequences based on estimating the 3D motion of individual objects and the camera.
SfM-Net predicts a batch of binary full image masks specyfing the object memberships of individual pixels with a standard encoder-decoder
network for pixel-wise prediction. A fully connected network branching off the encoder predicts a 3D motion for each object.
However, due to the fixed number of objects masks, the system can only predict a small number of motions and
often fails to properly segment the pixels into the correct masks or assigns background pixels to object motions.

Thus, this approach is very unlikely to scale to dynamic scenes with a potentially
large number of diverse objects due to the inflexible nature of their instance segmentation technique.

A scalable approach to instance segmentation based on region-based convolutional networks
was recently introduced with Mask R-CNN \cite{MaskRCNN}, which inherits the ability to detect
a large number of objects from a large number of classes at once from Faster R-CNN
and predicts pixel-precise segmentation masks for each detected object.

We propose \emph{Motion R-CNN}, which combines the scalable instance segmentation capabilities of
Mask R-CNN with the end-to-end 3D motion estimation approach introduced with SfM-Net.
For this, we naturally integrate 3D motion prediction for individual objects into the per-RoI R-CNN head
in parallel to classification and bounding box refinement.

\subsection{Related work}

\paragraph{Deep networks in optical flow and scene flow}

\cite{FlowLayers}
\cite{ESI}

\paragraph{Slanted plane methods for 3D scene flow}
The slanted plane model for scene flow \cite{PRSF, PRSM} models a 3D scene as being
composed of planar segments. Pixels are assigned to one of the planar segments,
each of which undergoes a rigid motion.

In contrast to \cite{PRSF, PRSM}, the Object Scene Flow method \cite{KITTI2015}
assigns each slanted plane to one rigidly moving object instance, thus
reducing the number of independently moving segments by allowing multiple
segments to share the motion of the object they belong to.

In a recent approach termed Instance Scene Flow \cite{InstanceSceneFlow},
a CNN is used to compute 2D bounding boxes and instance masks, which are then combined
with depth obtained from a non-learned stereo algorithm to be used as pre-computed
inputs to the object scene flow model from \cite{KITTI2015}.

Interestingly, these slanted plane methods achieve the current state-of-the-art
in scene flow \emph{and} optical flow estimation on the KITTI benchmarks \cite{KITTI2012, KITTI2015},
outperforming end-to-end deep networks like \cite{FlowNet2, SceneFlowDataset}.

%
In other contexts, the move from
% talk about performance issues with energy-minimization components, draw parallels to evolution of R-CNNs in terms of speed and accuracy when moving towards full end-to-end learning

\paragraph{End-to-end deep networks for 3D rigid motion estimation}
End-to-end deep learning for predicting rigid 3D object motions was first introduced with
SE3-Nets \cite{SE3Nets}, which take raw 3D point clouds as input and produce a segmentation
of the points into objects together with the 3D motion of each object.
Bringing this idea to the context of image sequences, SfM-Net \cite{SfmNet} takes two consecutive frames and
estimates a segmentation of pixels into objects together with their 3D motions between the frames.
In addition, SfM-Net predicts dense depth and camera motion to obtain full 3D scene flow from end-to-end deep learning.
For supervision, SfM-Net penalizes the dense optical flow composed from all 3D motions and the depth estimate
with a brightness constancy proxy loss.

\label{sec:introduction}

\section{Background}
\parindent 2em
\onehalfspacing

\label{sec:background}
In this section, we will give a more detailed description of previous works
we directly build on and other prerequisites.

\subsection{Basic definitions}
For regression, we define the smooth $\ell_1$-loss as
\begin{equation}
\ell_1^*(x) =
\begin{cases}
0.5x^2 &\text{if |x| < 1} \\
|x| - 0.5 &\text{otherwise,}
\end{cases}
\end{equation}
which provides a certain robustness to outliers and will be used
frequently in the following chapters.
For classification we define the cross-entropy loss as
\begin{equation}
\ell_{cls} =
\end{equation}

\subsection{Optical flow and scene flow}
Let $I_1,I_2 : P \to \mathbb{R}^3$ be two temporally consecutive frames in a
sequence of images.
The optical flow
$\mathbf{w} = (u, v)^T$ from $I_1$ to $I_2$
maps pixel coordinates in the first frame $I_1$ to pixel coordinates of the
visually corresponding pixel in the second frame $I_2$,
and can be interpreted as the apparent movement of brigthness patterns between the two frames.
Optical flow can be regarded as two-dimensional motion estimation.

Scene flow is the generalization of optical flow to 3-dimensional space and
requires estimating depth for each pixel. Generally, stereo input is used for scene flow
to estimate disparity-based depth, however monocular depth estimation with deep networks is becoming
popular \cite{DeeperDepth}.

\subsection{Convolutional neural networks for dense motion estimation}
Deep convolutional neural network (CNN) architectures
\cite{ImageNetCNN, VGGNet, ResNet}
became widely popular through numerous successes in classification and recognition tasks.
The general structure of a CNN consists of a convolutional encoder, which
learns a spatially compressed, wide (in the number of channels) representation of the input image,
and a fully connected prediction network on top of the encoder.

The compressed representations learned by CNNs of these categories do not, however, allow
for prediction of high-resolution output, as spatial detail is lost through sequential applications
of pooling or strides.
Thus, networks for dense prediction introduce a convolutional decoder on top of the representation encoder,
performing upsampling of the compressed features and resulting in a encoder-decoder pyramid.
The most popular deep networks of this kind for end-to-end optical flow prediction
are variants of the FlowNet family \cite{FlowNet, FlowNet2},
which was recently extended to scene flow estimation \cite{SceneFlowDataset}.
Figure \ref{} shows the classical FlowNetS architecture for optical fow prediction.
Note that the network itself is a rather generic autoencoder and is specialized for optical flow only through being trained
with supervision from dense optical flow ground truth.
Potentially, the same network could also be used for semantic segmentation if
the number of output channels was adapted from two to the number of classes. % TODO verify
Still, FlowNetS demonstrates that a generic deep encoder-decoder CNN can learn to perform image matching arguably well,
given just two consecutive frames as input and a large enough receptive field at the outputs to cover the displacements.
Note that the maximum displacement that can be correctly estimated only depends on the number of 2D strides or pooling
operations in the encoder.
Recently, other encoder-decoder CNNs have been applied to optical flow as well \cite{DenseNetDenseFlow}.

\subsection{SfM-Net}
Here, we will describe the SfM-Net architecture in more detail and show their results
and some of the issues.

\subsection{ResNet}
\label{ssec:resnet}
For completeness, we will give a short review of the ResNet \cite{ResNet} architecture we will use
as a backbone CNN for our network.

\subsection{Region-based convolutional networks}
\label{ssec:rcnn}
We now give a short review of region-based convolutional networks, which are currently by far the
most popular deep networks for object detection, and have recently also been applied to instance segmentation.

\paragraph{R-CNN}
Region-based convolutional networks (R-CNNs) \cite{RCNN} use a non-learned algorithm external to a standard encoder CNN
for computing \emph{region proposals} in the shape of 2D bounding boxes, which represent regions that may contain an object.
For each of the region proposals, the input image is cropped using the regions bounding box and the crop is
passed through a CNN, which performs classification of the object (or non-object, if the region shows background).

\paragraph{Fast R-CNN}
The original R-CNN involves computing one forward pass of the CNN for each of the region proposals,
which is costly, as there is generally a large number of proposals.
Fast R-CNN \cite{FastRCNN} significantly reduces computation by performing only a single forward pass with the whole image
as input to the CNN (compared to the sequential input of crops in the case of R-CNN).
Then, fixed size crops are taken from the compressed feature map of the image,
each corresponding to one of the proposal bounding boxes.
The crops are collected into a batch and passed into a small Fast R-CNN
\emph{head} network, which performs classification and prediction of refined boxes for all regions in one forward pass.
This technique is called \emph{RoI pooling}. % TODO explain how RoI pooling converts full image box coords to crop ranges
\todo{more details and figure}
Thus, given region proposals, the per-region computation is reduced to a single pass through the complete network,
speeding up the system by orders of magnitude. % TODO verify that

\paragraph{Faster R-CNN}
After streamlining the CNN components, Fast R-CNN is limited by the speed of the region proposal
algorithm, which has to be run prior to the network passes and makes up a large portion of the total
processing time.
The Faster R-CNN object detection system \cite{FasterRCNN} unifies the generation of region proposals and subsequent box refinement and
classification into a single deep network, leading to faster processing when compared to Fast R-CNN
and again, improved accuracy.
This unified network operates in two stages.
In the \emph{first stage}, one forward pass is performed on the \emph{backbone} network,
which is a deep feature encoder CNN with the original image as input.
Next, the \emph{backbone} output features are passed into a small, fully convolutional \emph{Region Proposal Network (RPN)} head, which
predicts objectness scores and regresses bounding boxes at each of its output positions.
At any position, bounding boxes are predicted as offsets relative to a fixed set of \emph{anchors} with different
aspect ratios.
\todo{more details and figure}
% TODO more about striding & computing the anchors?
For each anchor at a given position, the objectness score tells us how likely this anchors is to correspond to a detection.
The region proposals can then be obtained as the N highest scoring anchor boxes.

The \emph{second stage} corresponds to the original Fast R-CNN head network, performing classification
and bounding box refinement for each region proposal. % TODO verify that it isn't modified
As in Fast R-CNN, RoI pooling is used to crop one fixed size feature map for each of the region proposals.

\paragraph{Feature Pyramid Networks}
In Faster R-CNN, a single feature map is used as a source of all RoIs, independent
of the size of the bounding box of the RoI.
However, for small objects, the C4 \todo{explain terminology of layers} features
might have lost too much spatial information to properly predict the exact bounding
box and a high resolution mask. Likewise, for very big objects, the fixed size
RoI window might be too small to cover the region of the feature map containing
information for this object.
As a solution to this, the Feature Pyramid Network (FPN) \cite{FPN} enable features
of an appropriate scale to be used, depending of the size of the bounding box.
For this, a pyramid of feature maps is created on top of the ResNet \cite{ResNet}
encoder. \todo{figure and more details}
Now, during RoI pooling,
\todo{show formula}.


\paragraph{Mask R-CNN}
Faster R-CNN and the earlier systems detect and classify objects at bounding box granularity.
However, it can be helpful to know class and object (instance) membership of all individual pixels,
which generally involves computing a binary mask for each object instance specifying which pixels belong
to that object. This problem is called \emph{instance segmentation}.
Mask R-CNN \cite{MaskRCNN} extends the Faster R-CNN system to instance segmentation by predicting
fixed resolution instance masks within the bounding boxes of each detected object.
This is done by simply extending the Faster R-CNN head with multiple convolutions, which
compute a pixel-precise mask for each instance.
In addition to extending the original Faster R-CNN head, Mask R-CNN also introduced a network
variant based on Feature Pyramid Networks \cite{FPN}.
Figure \ref{} compares the two Mask R-CNN head variants.
\todo{RoI Align}

\paragraph{Bounding box regression}
All bounding boxes predicted by the RoI head or RPN are estimated as offsets
with respect to a reference bounding box. In the case of the RPN,
the reference bounding box is one of the anchors, and refined bounding boxes from the RoI head are
predicted relative to the RPN output bounding boxes.
Let $(x, y, w, h)$ be the top left coordinates, height and width of the bounding box
to be predicted. Likewise, let $(x^*, y^*, w^*, h^*)$ be the ground truth bounding
box and let $(x_r, y_r, w_r, h_r)$ be the reference bounding box.
We then define the ground truth \emph{box encoding} $b^*$ as
\begin{equation*}
b^* = (b_x^*, b_y^*, b_w^*, b_h^*),
\end{equation*}
where
\begin{equation*}
b_x^* = \frac{x^* - x_r}{w_r},
\end{equation*}
\begin{equation*}
b_y^* = \frac{y^* - y_r}{h_r}
\end{equation*}
\begin{equation*}
b_w^* = \log \left( \frac{w^*}{w_r} \right)
\end{equation*}
\begin{equation*}
b_h^* = \log \left( \frac{h^*}{h_r} \right),
\end{equation*}
which represents the regression target for the bounding box refinement
outputs of the network.

In the same way, we define the predicted box encoding $b$ as
\begin{equation*}
(b_x, b_y, b_w, b_h),
\end{equation*}
where
\begin{equation*}
b_x = \frac{x - x_r}{w_r},
\end{equation*}
\begin{equation*}
b_y = \frac{y - y_r}{h_r}
\end{equation*}
\begin{equation*}
b_w = \log \left( \frac{w}{w_r} \right)
\end{equation*}
\begin{equation*}
b_h = \log \left( \frac{h}{h_r} \right).
\end{equation*}

At test time, to get from a predicted box encoding $(b_x, b_y, b_w, b_h)$ to the actual bounding box $(x, y, w, h)$,
we invert the definitions above,
\begin{equation*}
x = b_x \cdot w_r + x_r,
\end{equation*}
\begin{equation*}
y = b_y \cdot b_r + y_r,
\end{equation*}
\begin{equation*}
w = \exp(b_w) \cdot w_r,
\end{equation*}
\begin{equation*}
h = \exp(b_h) \cdot h_r,
\end{equation*}
and thus obtain the bounding box as the reference bounding box adjusted by
the predicted relative offsets and scales.


\paragraph{Supervision of the RPN}
\todo{TODO}

\paragraph{Supervision of the RoI head}
\todo{TODO}


\section{Motion R-CNN}
\parindent 2em
\onehalfspacing

\label{sec:approach}

\subsection{Motion R-CNN architecture}
\subsection{Supervision}
%\subsection{Per-RoI motion loss}
\subsection{Dense flow from instance-level prediction}


\section{Experiments}
\parindent 2em
\onehalfspacing


\subsection{Datasets}

\paragraph{Virtual KITTI}
The synthetic Virtual KITTI dataset is a re-creation of the KITTI driving scenario,
rendered from virtual 3D street scenes.
The dataset is made up of a total of 2126 frames from five different monocular sequences recorded from a camera mounted on
a virtual car.
Each sequence is rendered with varying lighting and weather conditions and from different viewing angles, resulting
in a total of 10 variants per sequence.
In addition to the RGB frames, a variety of ground truth is supplied.
For each frame, we are given a dense depth and optical flow map,
2D and 3D object bounding boxes, instance masks and 3D poses of all cars and vans in the scene,
the camera extrinsics matrix, and various other labels.

This makes the Virtual KITTI dataset ideally suited for developing our joint instance segmentation
and motion estimation system, as it allows us to test different components in isolation and
progress to more and more complete predictions.

\subsection{Training Setup}
Our training schedule is similar to the Mask R-CNN Cityscapes schedule.
We train on a single Titan X (Pascal) for a total of 192K iterations.
As learning rate we use $0.25 \cdot 10^{-2}$ for the first 144K iterations and $0.25 \cdot 10^{-3}$
for all remaining iterations.

\subsection{Experiments on Virtual KITTI}

\subsection{Evaluation on KITTI 2015}

\label{sec:experiments}

\section{Conclusion}
\parindent 2em
\onehalfspacing

We have introduced an extension on top of region-based convolutional networks to enable object motion estimation
in parallel to instance segmentation.
\todo{complete}

\subsection{Future Work}
\paragraph{Predicting depth}
In most cases, we want to work with raw RGB sequences for which no depth is available.
To do so, we could integrate depth prediction into our network by branching off a
depth network from the backbone in parallel to the RPN, as in Figure \ref{}.
Although single-frame monocular depth prediction with deep networks was already done
to some level of success,
our two-frame input should allow the network to make use of epipolar
geometry for making a more reliable depth estimate, at least when the camera
is moving.

\paragraph{Training on real world data}
Due to the amount of supervision required by the different components of the network
and the complexity of the optimization problem,
we trained Motion R-CNN on the simple synthetic Virtual KITTI dataset for now.
A next step will be training on a more realistic dataset.
For this, we can first pre-train the RPN on an object detection dataset like
Cityscapes. As soon as the RPN works reliably, we could execute alternating
steps of training on, for example, Cityscapes and the KITTI stereo and optical flow datasets.
On KITTI stereo and flow, we could run the instance segmentation component in testing mode and only penalize
the motion losses (and depth prediction if added), as no instance segmentation ground truth exists.
On Cityscapes, we could continue train the instance segmentation components to
improve detection and masks and avoid forgetting instance segmentation.
As an alternative to this training scheme, we could investigate training on a pure
instance segmentation dataset with unsupervised warping-based proxy losses for the motion (and depth) prediction.

\label{sec:conclusion}

%%%%%%%%%%%%%%%%%%%%%%%%%%%%%%%%%%%%%%%%%%%%%%%%%%
% Bibliografie mit BibLaTeX
%
% Verwende keyword=meinbegriff, um nur die Einträge aus deiner .bib-Datei ausgeben zu lassen, die mit meinbegriff getaggt sind.
% Darf ein bestimmtes Keyword nicht enthalten sein, verwende notkeyword=meinbegriff.
\singlespacing
\printbibliography[title=Bibliography, heading=bibliography]
%\printbibliography[title=Literaturverzeichnis, heading=bibliography, keyword=meinbegriff]


\clearpage
%%%%%%%%%%%%%%%%%%%%%%%%%%%%%%%%%%%%%%%%%%%%%%%%%%
% Beginn des Anhangs
\appendix

%% \pagenumbering{Roman}
%% \setcounter{page}{\value{savedromanpagenumber}+1} % gespeicherte Seitenzahl vom Beginn des Dokuments aufrufen und damit weiterzählen

\clearpage

\end{document}
