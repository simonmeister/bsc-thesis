
\subsection{Motion R-CNN architecture}

Building on Mask R-CNN, we enable per-object motion estimation by predicting the 3D motion of each detected object
in camera space.

\paragraph{Backbone Network}
Like Faster R-CNN and Mask R-CNN, we use a ResNet variant as backbone network to compute feature maps from input imagery.

Inspired by FlowNetS, we make one modification to enable image matching within the backbone network,
laying the foundation for our motion estimation. Instead of taking a single image as input to the backbone,
we depth-concatenate two temporally consecutive frames $I_t$ and $I_{t+1}$, yielding a input image map with six channels.
We do not introduce a separate network for computing region proposals and use our modified backbone network
as both first stage RPN and second stage feature extractor for region cropping.

\paragraph{Per-RoI motion prediction}
We use a rigid 3D motion parametrization similar to the one used by SfM-Net and SE3-Nets \cite{SfmNet,SE3Nets}.
For the $k$-th object proposal, we predict the rigid transformation $\{R_t^k, t_t^k\}\in \mathbf{SE}(3)$
\footnote{$\mathbf{SE}(3)$ refers to the Special Euclidean Group representing 3D rotations
and translations: $\{R, t|R \in \mathbf{SO}(3), t \in \mathbb{R}^3\}$}
of the object between the two frames $I_t$ and $I_{t+1}$ as well as the object pivot $p_t^k \in \mathbb{R}^3$ at time $t$ .
We parametrize ${R_t^k}$ using an Euler angle representation,

\begin{equation}
R_t^k = R_t^{k,z}(\gamma) \cdot R_t^{k,x}(\alpha) \cdot R_t^{k,y}(\beta),
\end{equation}

where
\begin{equation}
R_t^{k,x}(\alpha) =
\begin{pmatrix}
  1 & 0 & 0 \\
  0 & \cos(\alpha) & -\sin(\alpha) \\
  0 & \sin(\alpha) & \cos(\alpha)
\end{pmatrix},
\end{equation}

\begin{equation}
R_t^{k,y}(\beta) =
\begin{pmatrix}
  \cos(\beta) & 0 & \sin(\beta) \\
  0 & 1 & 0 \\
  -\sin(\beta) & 0 & \cos(\beta)
\end{pmatrix},
\end{equation}

\begin{equation}
R_t^{k,z}(\gamma) =
\begin{pmatrix}
  \cos(\gamma) & -\sin(\gamma) & 0 \\
  \sin(\gamma) & \cos(\gamma) & 0 \\
  0 & 0 & 1
\end{pmatrix},
\end{equation}

and $\alpha, \beta, \gamma$ are the rotation angles about the $x,y,z$-axis, respectively.


Figure \ref{fig:motion_rcnn_head} shows our extended per-RoI head network.
We then extend the Faster R-CNN head by adding a fully connected layer in parallel to the final fully connected layers for
predicting refined boxes and classes.
Like for refined boxes and masks, we make one separate motion prediction for each class.
Each motion is predicted as a set of nine scalar motion parameters,
$\sin(\alpha)$, $\sin(\beta)$, $\sin(\gamma)$, $t_t^k$ and $p_t^k$,
where $\sin(\alpha)$, $\sin(\beta)$ and $\sin(\gamma)$ are clipped to $[-1, 1]$.
Here, we assume that motions between frames are relatively small
and that objects rotate at most 90 degrees in either direction along any axis.

\paragraph{Camera motion prediction}
In addition to the object transformations, we optionally predict the camera motion $\{R_t^{cam}, t_t^{cam}\}\in \mathbf{SE}(3)$
between the two frames $I_t$ and $I_{t+1}$.
For this, we flatten the full output of the backbone and pass it through a fully connected layer.
We again represent $R_t^{cam}$ using a Euler angle representation and
predict $\sin(\alpha)$, $\sin(\beta)$, $\sin(\gamma)$ and $t_t^{cam}$ in the same way as for the individual objects.

\subsection{Supervision}

\paragraph{Per-RoI supervision with motion ground truth}
Given a positive RoI, let $i_k$ be the index of the matched ground truth example with class $c_k$,
let $R_{c_k}^k, t_{c_k}^k, p_{c_k}^k$ be the predicted motion for class $c_k$
and $R_{gt}^{i_k}, t_{gt}^{i_k}, p_{gt}^{i_k}$ the ground truth motion for the example $i_k$.
We compute the motion loss $L_{motion}^k$ for each RoI as

\begin{equation}
L_{motion}^k =l_{R}^k + l_{t}^k + l_{p}^k,
\end{equation}
where
\begin{equation}
l_{R}^k = \arccos\left( \min\left\{1, \max\left\{-1, \frac{tr(inv(R_{c_k}^k) \cdot R_{gt}^{i_k}) - 1}{2} \right\}\right\} \right)
\end{equation}
measures the angle of the error rotation between predicted and ground truth rotation,
\begin{equation}
l_{t}^k = \lVert inv(R_{c_k}^k) \cdot (t_{gt}^{i_k} - t_{c_k}^k) \rVert,
\end{equation}
is the euclidean norm between predicted and ground truth translation, and
\begin{equation}
l_{p}^k = \lVert p_{gt}^{i_k} - p_{c_k}^k \rVert
\end{equation}
is the euclidean norm between predicted and ground truth pivot.


\subsection{Dense flow from motion}
We compose a dense optical flow map from the outputs of our Motion R-CNN network.
Given the depth map $d_t$ for frame $I_t$, we first create a 3D point cloud in camera space at time $t$,
where
\begin{equation}
P_t =
\begin{pmatrix}
X_t \\ Y_t \\ Z_t
\end{pmatrix}
=
\frac{d_t}{f}
\begin{pmatrix}
x_t - c_0 \\ y_t - c_1 \\ f
\end{pmatrix},
\end{equation}
is the 3D coordinate at $t$ corresponding to the point with pixel coordinates $x_t, y_t$,
which range over all coordinates in $I_t$.

Given $k$ detections with predicted motions as above, we transform all points within the bounding
box of a detected object according to the predicted motion of the object.

We first define the \emph{full image} mask $m_t^k$ for object k,
which can be computed from the predicted box mask $m_k^b$ by bilinearly resizing
$m_k^b$ to the width and height of the predicted bounding box and then copying the values
of the resized mask into a full image map starting at the top-right coordinate of the predicted bounding box.
Then,
\begin{equation}
P'_{t+1} =
P_t + \sum_1^{k} m_t^k\left\{ R_t^k \cdot (P_t - p_t^k) + p_t^k + t_t^k - P_t \right\}
\end{equation}

Next, we transform all points given the camera transformation $\{R_t^c, t_t^c\} \in SE3$, % TODO introduce!

\begin{equation}
\begin{pmatrix}
X_{t+1} \\ Y_{t+1} \\ Z_{t+1}
\end{pmatrix}
= P_{t+1} = R_t^c \cdot P'_{t+1} + t_t^k
\end{equation}.

Finally, we project the transformed 3D points at time $t+1$ to pixel coordinates again,
\begin{equation}
\begin{pmatrix}
x_{t+1} \\ y_{t+1}
\end{pmatrix}
=
\frac{f}{Z_{t+1}}
\begin{pmatrix}
X_{t+1} \\ Y_{t+1}
\end{pmatrix}
+
\begin{pmatrix}
c_0 \\ c_1
\end{pmatrix}.
\end{equation}
We can now obtain the optical flow between $I_t$ and $I_{t+1}$ at each point as
\begin{equation}
\begin{pmatrix}
u \\ v
\end{pmatrix}
=
\begin{pmatrix}
x_{t+1} - x_{t} \\ y_{t+1} - y_{t}
\end{pmatrix}.
\end{equation}


%Given the predicted motion as above, a depth map $d_t$ for frame $I_t$ and
%the predicted or ground truth camera motion $\{R_c^k, t_c^k\}\in SE3$.
