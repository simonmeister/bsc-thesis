
\subsection{Problem formulation}
\subsection{Object detection, semantic segmentation and instance segmentation}

\subsection{Optical flow, scene flow and structure from motion}
Let $I_1,I_2 : P \to \mathbb{R}^3$ be two temporally consecutive frames in a sequence of images.
The optical flow $\mathbf{w} = (u, v)^T$ from $I_1$ to $I_2$ maps pixel coordinates in the first
frame $I_1$ to pixel coordinates of the visually corresponding pixel in the second frame $I_2$, thus
representing the apparent movement of brigthness patterns between the two frames.
Optical flow can be regarded as two-dimensional motion estimation.

Scene flow is the generalization of optical flow to 3-dimensional space.

\subsection{Rigid scene model}
\subsection{Convolutional neural networks for dense estimation tasks}
Deep convolutional neural network (CNN) architectures \cite{} became widely popular
through numerous successes in classification and recognition tasks.
The general structure of a CNN consists of a convolutional encoder, which
learns a spatially compressed, wide (in the number of channels) representation of the input image,
and a fully connected prediction network on top of the encoder.

The compressed representations learned by CNNs of these categories do not, however, allow
for prediction of high-resolution output, as spatial detail is lost through sequential applications
of pooling or strides.
Thus, networks for dense prediction introduced a convolutional decoder in addition to the representation encoder,
performing upsampling of the compressed features and resulting in a encoder-decoder pyramid.
The most popular deep architecture of this kind for end-to-end optical flow prediction
is the FlowNet family of networs \cite{}, which was recently extended to scene flow estimation \cite{}.

% The conclusion should be an understanding of the generic nature of the popular dense prediction networks
% for flow and depth, which primarily stems from the fact that they are quick re-purposing of recognition CNNs.

% The reader should understand the limitations of the generic dense-estimator approach!

% Also, it should be emphasized that FlowNet learns to match images with a generic encoder,
% thus motivating the introduction of our motion head, which should integrate (and regularize) matching information learned
% in the resnet backbone.

\subsection{Region-based convolutional networks}
In the following, we will quickly re-view region-based convolutional networks, which are now the standard deep architecture for
object detection, object recognition and instance segmentation.

\paragraph{R-CNN}
Region-based convolutional networks (R-CNNs) use a non-learned algorithm external to a standard encoder CNN
for computing \emph{region proposals} in the shape of 2D bounding boxes, which represent regions that may contain an object.
Then, for each of the region proposals, the image is cropped at the proposed region and the crop is
passed through a CNN, which performs classification of the object (or non-object, if the region shows background).

\paragraph{Fast R-CNN}
The original R-CNN involved computing on forward pass of the CNN for each of the region proposals,
which can be costly, as there may be a large amount of proposals.
Fast R-CNN significantly reduces processing time by performing only a single forward pass with the whole image
as input to the CNN (compared to the input of crops in the case of R-CNN).
Then, crops are taken from the compressed feature map of the image, collected into a batch and passed into a small Fast R-CNN
\emph{head} network.
Thus, the per-region computation is heavily reduced, speeding up the system by orders of magnitude. % TODO verify that
This technique is called \emph{RoI pooling}.

\paragraph{Faster R-CNN: End-to-end deep object detection with }
The Faster-RCNN object detection system combines the generation of region proposals and subsequent box refinement and
classification into a single deep network, leading to faster processing when compared to Fast R-CNN
and again, improved accuracy.


\paragraph{Mask R-CNN}

Combining object detection and semantic segmentation, Mask R-CNN extends the Faster R-CNN system to predict
high resolution instance masks within the bounding boxes of each detected object.
This can be done by simply extending the Faster R-CNN head with multiple convolutions, which
compute a pixel-precise mask for each instance.
In addition, Mask R-CNN
