
\subsection{Datasets}

\paragraph{Virtual KITTI}
The synthetic Virtual KITTI dataset is a re-creation of the KITTI driving scenario,
rendered from virtual 3D street scenes.
The dataset is made up of a total of 2126 frames from five different monocular sequences recorded from a camera mounted on
a virtual car.
Each sequence is rendered with varying lighting and weather conditions and from different viewing angles, resulting
in a total of 10 variants per sequence.
In addition to the RGB frames, a variety of ground truth is supplied.
For each frame, we are given a dense depth and optical flow map,
2D and 3D object bounding boxes, instance masks and 3D poses of all cars and vans in the scene,
the camera extrinsics matrix, and various other labels.

This makes the Virtual KITTI dataset ideally suited for developing our joint instance segmentation
and motion estimation system, as it allows us to test different components in isolation and
progress to more and more complete predictions.

\subsection{Training Setup}
Our training schedule is similar to the Mask R-CNN Cityscapes schedule.
We train on a single Titan X (Pascal) for a total of 192K iterations.
As learning rate we use $0.25 \cdot 10^{-2}$ for the first 144K iterations and $0.25 \cdot 10^{-3}$
for all remaining iterations.

\subsection{Experiments on Virtual KITTI}

\subsection{Evaluation on KITTI 2015}
